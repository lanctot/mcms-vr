\documentclass[jair,twoside,11pt,theapa]{article}
\usepackage{jair, theapa, rawfonts}

% uncomment later
%\jairheading{1}{1993}{1-15}{6/91}{9/91}
%\ShortHeadings{Minimizing Conflicts: A Heuristic Repair Method}
%{Minton, Philips, Johnston, \& Laird}
%\firstpageno{25}

\begin{document}

\title{Monte Carlo Search Algorithms for Two-Player Games with Chance Nodes}

\author{}

%\author{\name Steven Minton \email minton@ptolemy.arc.nasa.gov \\
%       \name Andy Philips \email philips@ptolemy.arc.nasa.gov \\
%       \addr NASA Ames Research Center, Mail Stop: 244-7,\\
%       Moffett Field, CA  94035 USA
%       \AND
%       \name Mark D. Johnston \email johnston@stsci.edu \\
%      \addr Space Telescope Science Institute,
%       3700 San Martin Drive,\\
%       Baltimore, MD 21218 USA
%       \AND
%       \name Philip Laird \email laird@ptolemy.arc.nasa.gov \\
%       \addr NASA Ames Research Center,
%       AI Research Branch, Mail Stop: 269-2,\\
%       Moffett Field, CA  94035 USA}

% For research notes, remove the comment character in the line below.
% \researchnote

\maketitle

\begin{abstract}
Over the past several years, Monte Carlo search techniques have gained in popularity 
due to their ability to reduce computational load by efficiently estimating values 
of subtrees.  
There has been a focus on analysis of these search techniques on deterministic
games, such as Go, Hex, and Lines of Action. 
This paper presents a new way of sampling in two-player games with chance events, 
such as Backgammon, adapted from sparse sampling techniques used in MDPs. 
We introduce Monte Carlo *-Minimax Search (MCMS), an extension of the Ballard's 
classic *-minimax algorithm that uses sparse sampling. 
We present a thorough theoretical analysis of MCMS, giving a rate of convergence to the
optimal strategy that does not depend on the number of states. 
We show that sparse sample can be used in Monte Carlo Tree Search (MCTS), and compare it
to a similar technique called double-progressive widening.
We also present variance reduction techniques and show how they can be applied in both 
MCMS and MCTS. 
We perform a thorough empirical evaluation on seven domains: 
Pig, EinStein w\"{u}rfelt nicht!, Can't Stop, Ra, Dominion, Carcassonne, and Backgammon.
\end{abstract}

\section{Introduction}
\label{sec:intro}

Bla bla~\cite{Lanctot13MCMS}


%\acks{The authors wish to thank Hans-Martin Adorf, Don Rosenthal, 
%Richard Franier, Peter Cheeseman and Monte Zweben for their assistance
%and advice.  We also thank Ron Musick and our anonymous reviewers for
%their comments.  The Space Telescope Science Institute is operated by
%the Association of Universities for Research in Astronomy for NASA.
%}

%\appendix
%\section*{Appendix A. Probability Distributions for N-Queens}

\bibliography{paper}
\bibliographystyle{theapa}

\end{document}






