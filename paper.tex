\documentclass[jair,twoside,11pt,theapa]{article}
\usepackage{jair, theapa, rawfonts}

% uncomment later
%\jairheading{1}{1993}{1-15}{6/91}{9/91}
%\ShortHeadings{Minimizing Conflicts: A Heuristic Repair Method}
%{Minton, Philips, Johnston, \& Laird}
%\firstpageno{25}

\begin{document}

\title{Monte Carlo Search Techniques for Two-Player Perfect Information Games with Chance Nodes}

\author{Authors}

%\author{\name Steven Minton \email minton@ptolemy.arc.nasa.gov \\
%       \name Andy Philips \email philips@ptolemy.arc.nasa.gov \\
%       \addr NASA Ames Research Center, Mail Stop: 244-7,\\
%       Moffett Field, CA  94035 USA
%       \AND
%       \name Mark D. Johnston \email johnston@stsci.edu \\
%      \addr Space Telescope Science Institute,
%       3700 San Martin Drive,\\
%       Baltimore, MD 21218 USA
%       \AND
%       \name Philip Laird \email laird@ptolemy.arc.nasa.gov \\
%       \addr NASA Ames Research Center,
%       AI Research Branch, Mail Stop: 269-2,\\
%       Moffett Field, CA  94035 USA}

% For research notes, remove the comment character in the line below.
% \researchnote

%- Journal (JAIR?) paper on Monte Carlo search in perfect information with chance nodes
%  - General aim is for the paper to be on modern Monte Carlo techniques for perfect information games with chance (PIC) nodes
%
%  - Main content overview: 
%    - combine ideas of sparse sampling and variance reduction in mcts + new stuff
%    - do so in a way that tells a complete story about modern Monte Carlo in PIC games
%
%  - JAIR asks for 20% new stuff: How is paper different from the MCMS and variance reduction papers? 
%    Here's a candidate list:
%
%    - theoretical improvements based on Karnin13, http://jmlr.org/proceedings/papers/v28/karnin13.pdf
%    - extend variance reduction to two-player setting
%    - mcts-solver for PIC games (Marc's "implicit minimax" idea, which is essentially the same but using heuristics) 
%      - implicit minimax can also be used with variance reduction
%    - comparison to BRUE? (seems like the right algorithm for games with chance, esp. with transpositions)
%    - classic alpha-beta heuristics
%    - dynamic widths in mcms
%    - add carcassonne and dominion to domain list? (backgammon.. probably bad idea)
%    - is there a synthetic tree model for games with chance?
%    - MCMS experiments with extending time length
%    - sampling without replacement: theory and practice
%    - Levi & Hendrik's ideas on stratified sampling for playouts

\maketitle

\begin{abstract}
Over the past several years, Monte Carlo search techniques have gained in popularity 
due to their ability to reduce computational load by efficiently estimating values 
of subtrees.  
There has been a focus on analysis of these search techniques on deterministic
games, such as Go, Hex, and Lines of Action. 
This paper presents a new way of sampling in two-player games with chance events, 
adapted from sparse sampling techniques used in MDPs. 
We describe Monte Carlo *-Minimax Search (MCMS), an extension of the Ballard's 
classic *-minimax algorithm that uses sparse sampling. 
We present a thorough theoretical analysis of MCMS, giving a rate of convergence to the
optimal strategy that does not depend on the number of states. 
We show that sparse sample can be used in the Monte Carlo Tree Search (MCTS) paradigm as well, 
and compare it to a similar technique called double-progressive widening.
We also present variance reduction techniques and show how they can be applied in both 
MCMS and MCTS. 
We perform a thorough empirical evaluation on several domains comparing the relative benefits 
and costs of the different Monte Carlo sampling techniques. 
\end{abstract}

\section{Introduction}
\label{sec:intro}

Perfect information games with chance. 
How and why are they different than games without chance?
Why are they interesting?
We will look at modern approaches and techniques for this class of games.
Give a highlight of what is new in this paper.   

\section{Background}

Notation and terminology.
Related word: Original sparse sampling in MDPs. AMS, FSSS.
Two major search paradigms: minimax and MCTS (how are they different? why do we care... for this paper?)
Basic variance reduction techniques. 

\section{Sparse Sampling}

Overview of how to adapt sparse sampling to adversarial search in both minimax and MCTS.

\subsection{Monte Carlo *-Minimax}

MCMS~\cite{Lanctot13MCMS}

\subsection{Monte Carlo Tree Search}

Point out problem with DPW. Abdallah's suggested fix. Describe MCTS-SS and Rotating MCTS-SS. Consistency and comparison to DPW?

\section{Variance Reduction Techniques}

In games of chance, especially Monte Carlo techniques, variance is a problem. Describe how we apply variance reduction in search.
Describe how to apply in minimax and MCTS. 

\section{Experimental Evaluation}

Thorough experimental evaluation.

\subsection{Conclusion}

%\acks{
% Acknowledgements  
%}

%\appendix
%\section*{Appendix A. Probability Distributions for N-Queens}

\bibliography{paper}
\bibliographystyle{theapa}

\end{document}






